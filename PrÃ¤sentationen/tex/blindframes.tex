\section{Überschrift auf Ebene 1 (section)} % ================================\

\subsection{Überschrift auf Ebene 2 (subsection)} % -------------------------\

\begin{frame}
    \frametitle{Folientitel}
    \framesubtitle{Folienuntertitel}
    \blindtext
\end{frame}

% ---------------------------------------------------------------------------/
% ============================================================================/

\section{Listen} % ===========================================================\

\begin{frame}{Listen}{Beispiel einer Liste (itemize)}
    \blinditemize
\end{frame}

\begin{frame}{Listen}{Beispiel einer Liste (3*itemize)}
    \begin{itemize}
        \item Erster Listenpunkt, Stufe 1
        \begin{itemize}
            \item Erster Listenpunkt, Stufe 2
            \begin{itemize}
                \item Erster Listenpunkt, Stufe 3
                \item Zweiter Listenpunkt, Stufe 3
            \end{itemize}
            \item Zweiter Listenpunkt, Stufe 2
        \end{itemize}
        \item Zweiter Listenpunkt, Stufe 1
    \end{itemize}
\end{frame}

\begin{frame}{Listen}{Beispiel einer Liste (enumerate)}
    \blindenumerate
\end{frame}

\begin{frame}{Listen}{Beispiel einer Liste (3*enumerate)}
    \begin{enumerate}
        \item Erster Listenpunkt, Stufe 1
        \begin{enumerate}
            \item Erster Listenpunkt, Stufe 2
            \begin{enumerate}
                \item Erster Listenpunkt, Stufe 3
                \item Zweiter Listenpunkt, Stufe 3
            \end{enumerate}
            \item Zweiter Listenpunkt, Stufe 2
        \end{enumerate}
        \item Zweiter Listenpunkt, Stufe 1
    \end{enumerate}
\end{frame}

\begin{frame}{Listen}{Beispiel einer Liste (description)}
    \blinddescription
\end{frame}

\begin{frame}{Listen}{Beispiel einer Liste (3*description)}
    \begin{description}
        \item[Erster] Listenpunkt, Stufe 1
        \begin{description}
            \item[Erster] Listenpunkt, Stufe 2
            \begin{description}
                \item[Erster] Listenpunkt, Stufe 3
                \item[Zweiter] Listenpunkt, Stufe 3
            \end{description}
            \item[Zweiter] Listenpunkt, Stufe 2
        \end{description}
        \item[Zweiter] Listenpunkt, Stufe 1
    \end{description}
\end{frame}

% ============================================================================/

\section{Blöcke} % ===========================================================\

\begin{frame}{Blöcke}{Block, Alert-Block und Example-Block}
    \begin{block}{Block}
    Dies ist der Inhalt eines \verb"block" mit der Bezeichnung \verb"Block"
    \end{block}

    \begin{alertblock}{Wichtig}
    Dies ist der Inhalt eines \verb"alertblock" mit der Bezeichnung \verb"Wichtig"
    \end{alertblock}

    \begin{exampleblock}{Beispiel}
    Dies ist der Inhalt eines \verb"exampleblock" mit der Bezeichnung \verb"Beispiel"
    \end{exampleblock}
\end{frame}

\begin{frame}{Blöcke}{Theorem, Proof und Example}
    \begin{theorem}
    Dies ist der Inhalt eines \verb"theorem" Blocks
    \end{theorem}

    \begin{proof}
    Dies ist der Inhalt eines \verb"proof" Blocks
    \end{proof}

    \begin{example}
    Dies ist der Inhalt eines \verb"example" Blocks
    \end{example}
\end{frame}

% ============================================================================/

\section{Zitate, Formeln, Tabellen und Abbildungen} % ========================\

\begin{frame}{Beispiel eines Zitats und einer Formel}
    Erstmals wurde der Zusammenhang von elektrischer Spannung, elektrischem Strom und elektrischem Widerstand von Georg Simon Ohm beschrieben. \cite{Ohm.1827}

    \begin{equation}
        U = R \cdot I
    \end{equation}
\end{frame}

\begin{frame}{Beispiel einer Tabelle}
    \begin{table}[htbp]
        \centering
        \caption{Beispieltabelle}
        \label{tab:bsp-tabelle}
        \begin{tabular}{lll}
            \toprule
            \textbf{Formelzeichen} & \textbf{Name} & \textbf{SI-Einheit} \\
            \midrule
            $U$ & Elektrische Spannung & \si{\volt} \\
            $R$ & Elektrischer Widerstand & \si{\ohm} \\
            $I$ & Elektrischer Strom & \si{\ampere} \\
            \bottomrule
        \end{tabular}
    \end{table}
\end{frame}

\begin{frame}{Beispiel einer Abbildung}
    \begin{figure}[htbp]
        \centering
        \includegraphics[width=0.5\textwidth]{example-image} % dieses Bild ist Teil des mwe Pakets, welches auch ohne Befehl in der Präambel aufgerufen wird
        \caption{Beispielbild}
        \label{fig:bsp-bild}
    \end{figure}
\end{frame}

% ============================================================================/

% Auf dieser Folie wird der wichtige Inhalt \alert{hervorgehoben}, um dies kenntlich zu machen.

\section{Overlays} % =========================================================\

\begin{frame}{Overlays}{Overlay Specifications}
    Dies ist ein Beispieltext.

    \begin{itemize}
        \item<1-> Dieser Text ist sofort sichtbar.
        \item<2-3> Dieser Text ist mit der zweiten und bis zur dritten Einblendung sichtbar.
        \item<3> Dieser Text ist nur während der dritten Einblendung sichtbar.
        \item<4-> Dieser Text ist mit der vierten Einblendung sichtbar.
    \end{itemize}
\end{frame}

\begin{frame}{Overlays}{\texttt{pause} Makro}
    Auf dieser Folie \pause

    wird der Text \pause

    nach und nach \pause

    eingeblendet.
\end{frame}

\begin{frame}{Overlays}{\texttt{alert} Makro}
    Auf dieser Folie wird der wichtige Inhalt im zweiten Schritt \alert<2>{hervorgehoben}, um dies kenntlich zu machen.
\end{frame}

% ============================================================================/

\section{Notizen} % ==========================================================\

\begin{frame}[fragile]{Notizen}

    Zu dieser Folie werden Notizen, welche nur in der \enquote{Referentenansicht} bzw. auf den zusätzlichen Notizfolien sichtbar sind, erstellt. Sichtbar ist dies nur im Quellcode bzw. wenn \verb|\setbeameroption{show only notes}| oder \verb|\setbeameroption{show notes on second screen}| verwendet wird.

    \note[item]{Notiz 1}
    \note[item]{Notiz 2}
    \note{Notiz 3}

\end{frame}

\note[itemize]{
    \item Notiz 1
    \item Notiz 2
}

\note[enumerate]{
    \item Notiz 3
    \item Notiz 4
}

\note{Notiz 5}

% ============================================================================/
