%%% Präambel #######################################################
\usepackage[utf8]{inputenc}                                         % Anpassung der Eingabekodierung
\usepackage[T1]{fontenc}                                            % Standard-Paket für die Auswahl der Schriftarten Kodierung
\usepackage{scrhack}                                                % Paket um Warnungen/Fehler durch veraltete Funktionen bei der Verwendung von KOMA Skript zu beheben (hier: Warnung bei erstem frontmatter Kapitel nach Titelseite)
\usepackage[automark,headsepline]{scrlayer-scrpage}                 % weitere Anpassungen der Dokumentenklasse
\usepackage[onehalfspacing]{setspace}                               % Zeilenabstand
\usepackage[english,ngerman]{babel}                                 % Sprachpaket ("english" ist in diesem Fall nur für den Text im "Abstract" notwendig, kann also ggf. entfernt werden; letzte Sprache hat Priorität)
\usepackage{csquotes}                                               % verbesserter Syntax und Verhalten von Anführungszeichen (empfohlen für biblatex)
\usepackage[style=ieee,sorting=none,giveninits=true,backend=biber]{biblatex} % Literaturverzeichnis und -verweise
\DeclareFieldFormat{title}{\textsl{#1}}                             % Titel im Literaturverzeichnis in schräger statt kursiver Schrift
\DeclareFieldFormat{booktitle}{\textsl{#1}}                         % s. o.
\DeclareFieldFormat{booksubtitle}{\textsl{#1}}                      % s. o.
\DeclareNameAlias{author}{family-given}                             % Nachnamen vor Vornamen anzeigen (vorher: last-first)
\DeclareNameAlias{editor}{family-given}                             % s. o.
\DeclareNameAlias{translator}{family-given}                         % s. o.
\DeclareNameAlias{default}{family-given}                            % s. o.
% \setcounter{biburllcpenalty}{7000}                                  % work-around für unschön umgebrochene URLs im Literaturverzeichnis
% \setcounter{biburlucpenalty}{7000}                                  % s. o.
% \setcounter{biburlnumpenalty}{7000}                                 % s. o.
\usepackage{etoolbox}                                               % Erstellung eigener Dokumentoptionen
\usepackage{lmodern}                                                % Schriftartenpaket
\usepackage{amsmath,amsfonts,amssymb}                               % AMS Pakete
\numberwithin{figure}{chapter}                                      % Kapitelnummer in Abbildungsbeschriftung
\numberwithin{table}{chapter}                                       % Kapitelnummer in Tabellenbeschriftung
\numberwithin{equation}{chapter}                                    % Kapitelnummer in Formelbeschriftung
\usepackage{siunitx}                                                % automatische und korrekte Schreibweise der SI Einheiten
\sisetup{
    locale=DE,                                                      % setzt \cdot statt \times in Exponentialschreibweise und output-decimal-marker = {,}
    group-digits = integer,
    group-separator = {.},                                          % Punkt als 1000er-Trennzeichen
    group-minimum-digits = 4,                                       % schon ab 4 Stellen gruppieren
    per-mode = symbol-or-fraction,                                  % per-mode wird in Abhängigkeit der Umgebung gesetzt, bei Fehlern: [per-mode=fraction] manuell setzen oder bei Drehzahlen: [per-mode=reciprocal]
    % product-symbol = \cdot,                                         % Zeichen bei Produkten, default: \times (x)
    list-final-separator = { und },
    list-pair-separator = { und },
    % list-units = single,                                          % Einheit nur beim letzten Listenelement; kann zu Missverständnissen führen
    % range-units = single,                                         % Einheit nur bei der größeren Bereichsgrenze; kann zu Missverständnissen führen
    range-phrase = { bis }
}
\DeclareSIUnit\dekade{Dekade}
% \DeclareSIPrefix\kilo{k}{3}
% \DeclareSIPower\quartic\tothefourth{4}                              % Bsp.: \unit{\kilogram\tothefourth} oder \unit{\quartic\metre}
% \DeclareSIQualifier\peakpeak{PP}
\usepackage[siunitx,EFvoltages]{circuitikz}                         % Option "EFvoltages" zeigt Spannungspfeile in Richtung des el. magn. Felds (also von + nach -)
\usepackage{graphicx}                                               % Grafiken einbinden
% \graphicspath{{img/}}                                               % Pfad für Grafiken (hier: Ordner "img" im Verzeichnis des Hauptdokuments)
% \usepackage{svg}                                                    % SVG-Grafiken einbinden
% \svgpath{svg/}                                                      % Pfad für SVG-Grafiken (hier: Ordner "svg" im Verzeichnis des Hauptdokuments)
\usepackage{caption}                                                % Beschriftungen an Abbildungen, Tabellen, etc.
\usepackage{subcaption}                                             % mehrere Abbildungen als eine darstellen
\usepackage{rotating}                                               % gedrehte Grafiken und Tabellen
\usepackage{xcolor}                                                 % Farbdefinitionen
\definecolor{MATLABplot1}{HTML}{0072BD}                             % hier beispielhaft aktuelle Standardfarben in MATLAB Plots
\definecolor{MATLABplot2}{HTML}{D95319}                             % https://de.mathworks.com/help/matlab/ref/plot.html#btzitot-Color
\definecolor{MATLABplot3}{HTML}{EDB120}
\definecolor{MATLABplot4}{HTML}{7E2F8E}
\definecolor{MATLABplot5}{HTML}{77AC30}
\definecolor{MATLABplot6}{HTML}{4DBEEE}
\definecolor{MATLABplot7}{HTML}{A2142F}
\usepackage{booktabs}                                               % Tabellen in Buchdruckqualität (horizontale Linien)
% \usepackage{multirow}                                               % mehrzeilige Zellen
% \usepackage{colortbl}                                               % farbige Zellen
\usepackage{listings}                                               % Code-Listings mit automatischem Syntax-Highlighting
\definecolor{codegray}{rgb}{0.5,0.5,0.5}
% \definecolor{MATLABtext}{HTML}{000000}
\definecolor{MATLABbackground}{HTML}{FFFFFF}
\definecolor{MATLABkeywords}{HTML}{0000FF}
\definecolor{MATLABstrings}{HTML}{A020F0}
\definecolor{MATLABcomments}{HTML}{3C763D}
\lstdefinestyle{MATLAB}{
    backgroundcolor=\color{MATLABbackground},
    commentstyle=\color{MATLABcomments},
    keywordstyle=\color{MATLABkeywords},
    stringstyle=\color{MATLABstrings},
    % identifierstyle=\color{blue},% hier wird auch das e in bspw. 10e-3 erkannt
    numberstyle=\tiny\color{codegray},
    basicstyle=\ttfamily\footnotesize,
    breakatwhitespace=false,
    breaklines=true,
    captionpos=b,
    keepspaces=true,
    numbers=left,
    numbersep=5pt,
    showspaces=false,
    showstringspaces=false,
    showtabs=false,
    tabsize=4,
    frame=single
}
\lstset{literate=%
    {Ö}{{\"O}}1
    {Ä}{{\"A}}1
    {Ü}{{\"U}}1
    {ß}{{\ss}}1
    {ü}{{\"u}}1
    {ä}{{\"a}}1
    {ö}{{\"o}}1
}
% \lstset{literate={~}{{\textasciitilde}}1} % wenn benötigt, in obere Liste einfügen
% \lstset{literate={µ}{{$\mathrm{\mu}$}}1} % s. o.

% ^^========= weitere Pakete über dieser Zeile einfügen ==========^^
% * Paket hyperref (und bookmark) immer als letztes laden! Paket glossaries nach hyperref laden!
\usepackage[ngerman]{hyperref}                                      % Hyperlinks in PDF Datei
\usepackage{bookmark}                                               % Lesezeichen in PDF Datei
% "You must load the hyperref package before the glossaries package to ensure the hyperlinks work." [S. 124, glossaries-user.pdf]
\usepackage[acronym,nopostdot,nonumberlist,toc]{glossaries}         % Glossar und Abkürzungen; für Eintrag im Inhaltsverzeichnis Option "toc" hinzufügen; "automake" "nomain" ?
% Änderung des Darstellungsformats und Berechnung des breitesten Eintrags: siehe "User Manual for glossaries.sty v4.46", Seite 222
\renewcommand{\glossarypreamble}{%
    \glsfindwidesttoplevelname[\currentglossary]}
\setglossarystyle{alttree}
% Code für Änderung des Darstellungsformats und Berechnung des breitesten Eintrags bis hier
\makeglossaries
% oder:
% \usepackage[nopostdot,nonumberlist]{glossaries}
% \usepackage[nohyperlinks, printonlyused]{acronym}
% Formatierung: (Bsp., weitere unter: https://www.dickimaw-books.com/gallery/glossaries-styles/)
% \usepackage{glossary-longragged}
% \setglossarystyle{longragged}
% https://tex.stackexchange.com/questions/415275/flush-glossaries-super-style-to-left

% Anpassung der Kopf- und Fußzeilen ================================
% \clearpairofpagestyles
% \lohead{...}
% \rohead{...}
% \cfoot[...]{...}
% \pagestyle{scrheadings}

% Anpassung der automatischen Bezeichnungen ========================
\renewcaptionname{ngerman}\sectionautorefname{Abschnitt}
\renewcaptionname{ngerman}\subsectionautorefname{Abschnitt}
\renewcaptionname{ngerman}\subsubsectionautorefname{Abschnitt}

% Anpassung der Seitenränder auf der Titelseite (funktioniert nur für \maketitle und nicht titlepage Umgebung)
% \renewcommand*{\coverpagetopmargin}{0mm}
% \renewcommand*{\coverpageleftmargin}{0mm}
% \renewcommand*{\coverpagerightmargin}{0mm}
% \renewcommand*{\coverpagebottommargin}{0mm}

% Anpassung der PDF Eigenschaften ==================================
\AtEndPreamble{
    \hypersetup{
        pdftitle    = {\VarTitel},
        pdfsubject  = {\VarArtDerArbeit{} im Studiengang \VarStudiengang},
        pdfauthor   = {\VarVerfasserIn},
        % pdfkeywords = {Stichwort1, Stichwort2 ...},
        % pdfdisplaydoctitle = true,                                      % Dokumenttitel statt Dateiname im PDF Viewer anzeigen
        % pdfpagelayout=TwoPageRight,                                     % Voreinstellung Ansicht: "TwoColumnRight" kontinuierliches Scrollen, "TwoPageRight" seitenweises Scrollen
        bookmarksnumbered = true,                                       % Abschnittsnummerierung in Lesezeichen einbeziehen
        bookmarksopen = true,                                           % Lesezeichen aufgeklappt anzeigen
        % colorlinks = true,                                              % Links farbig darstellen
        hidelinks,                                                      % Textfarbe von Links: schwarz
        unicode=true
    }
}

% neue Befehle für Dokumentfelder ==================================
\def\VarLogoHsH{7}
\newcommand{\LogoHsH}[1]{
    \def\VarLogoHsH{#1}
}
\newcommand{\Fakultaet}[1]{
    \def\VarFakultaet{#1}
}
\newcommand{\Fachgebiet}[1]{
    \def\VarFachgebiet{#1}
}
\newcommand{\ArtDerArbeit}[1]{
    \def\VarArtDerArbeit{#1}
}
\newcommand{\Studiengang}[1]{
    \def\VarStudiengang{#1}
}
\newcommand{\Titel}[1]{
    \def\VarTitel{#1}
}
\newcommand{\Untertitel}[1]{
    \def\VarUntertitel{#1}
}
\newcommand{\VerfasserIn}[3][Verfasser:in]{
    \def\VarBezVerfasserIn{#1}%
    \def\VarVerfasserIn{#2}%
    \def\VarVerfasserInMatNr{#3}
}
\newcommand{\DatumBeginn}[2][Beginn]{
    \def\VarBezDatumBeginn{#1}%
    \def\VarDatumBeginn{#2}
}
\newcommand{\DatumAbgabe}[2][Abgabe]{
    \def\VarBezDatumAbgabe{#1}%
    \def\VarDatumAbgabe{#2}
}
\newtoggle{DatumLeerzeile}
\newcommand{\DatumLeerzeile}{
    \toggletrue{DatumLeerzeile}
}
\newtoggle{DatenAlsZeitraum}
\newcommand{\DatenAlsZeitraum}[1][Zeitraum]{
    \def\VarBezDatenZeitraum{#1}%
    \toggletrue{DatenAlsZeitraum}
}
\newcommand{\BetreuerInA}[2][Betreuer:in]{
    \def\VarBetreuerInA{#2}%
    \def\VarBezBetreuerInA{#1}
}
\newcommand{\BetreuerInB}[2][]{
    \def\VarBetreuerInB{#2}%
    \def\VarBezBetreuerInB{#1}
}
\newcommand{\BetreuerInC}[2][]{
    \def\VarBetreuerInC{#2}%
    \def\VarBezBetreuerInC{#1}
}
\newcommand{\DatumVersion}[1]{
    \def\VarDatumVersion{#1}
}
\newtoggle{BetreuerInLeerzeileAB}
\newcommand{\BetreuerInLeerzeileAB}{
    \toggletrue{BetreuerInLeerzeileAB}
}
\newcommand{\BetreuerInLeerzeileBC}{
    \toggletrue{BetreuerInLeerzeileBC}
}


% [alle] Indizes Steil Stellen (von Cube, HsH-Forum)
% https://betech-hsh.de/viewtopic.php?f=119&t=756
\def\subinrm#1{\sb{\mathrm{#1}}}
{\catcode`\_=13 \global\let_=\subinrm}
\mathcode`_="8000
% Macros definieren:
\def\upsubscripts{\catcode`\_=12 }
\def\normalsubscripts{\catcode`\_=8 }
% Macro benutzen:
\upsubscripts %Indizes Steil
