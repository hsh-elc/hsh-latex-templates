%%% Quellenangabe nach Bildern
% https://tex.stackexchange.com/questions/206921/how-to-add-source-information-at-bottom-right-corner-of-a-figure-in-beamer
\newcommand{\Quelle}[2][Quelle]{\par\hfill {\tiny #1:~#2}}%\itshape

%%% neuer Befehl für Fakultätsauswahl
\def\VarHsHFak{6}
\newcommand{\HsHFak}[1]{
    \def\VarHsHFak{#1}
}

%%% Färbung und Änderung der Aufzählungszeichen als neuer Befehl
% https://tex.stackexchange.com/questions/14319/beamer-change-individual-bullet-color-in-itemize-list
\newcommand\ipro{\item[\textcolor{green}{\raisebox{0.2ex}{$\blacktriangle$}}]}
\newcommand\icon{\item[\textcolor{red}{\raisebox{0.2ex}{$\blacktriangledown$}}]}
% funktioniert
\newcommand\itodo{\item[\textcolor{red}{\textbf{!}}]}

%%% kleines hochgestelltes C im Kreis, P im Kreis, R im Kreis, TM
\newcommand{\supscrC}{\textsuperscript{\tiny\textcopyright}} % kleines hochgestelltes C im Kreis
\newcommand{\supscrP}{\textsuperscript{\tiny\textcircledP}} % kleines hochgestelltes P im Kreis
\newcommand{\supscrR}{\textsuperscript{\tiny\textregistered}} % kleines hochgestelltes R im Kreis
\newcommand{\supscrTM}{\textsuperscript{\tiny\texttrademark}} % kleines hochgestelltes TM

% [alle] Indizes Steil Stellen (von Cube, HsH-Forum)
% https://betech-hsh.de/viewtopic.php?f=119&t=756
\def\subinrm#1{\sb{\mathrm{#1}}}
{\catcode`\_=13 \global\let_=\subinrm}
\mathcode`_="8000
% Macros definieren:
\def\upsubscripts{\catcode`\_=12 }
\def\normalsubscripts{\catcode`\_=8 }
% Macro benutzen:
\upsubscripts %Indizes Steil
