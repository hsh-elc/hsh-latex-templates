\chapter{Hauptteil}
\label{chap:Hauptteil}

\section{Beispiele für Zitate, Formeln, Tabellen und Abbildungen}

Erstmals wurde der Zusammenhang von elektrischer Spannung, elektrischem Strom und elektrischem Widerstand von Georg Simon Ohm beschrieben. \cite{Ohm.1827}

\begin{equation}
    U = R \cdot I
\end{equation}

\begin{table}[htbp]
    \centering
    \caption{Beispieltabelle}
    \label{tab:bsp-tabelle}
    \begin{tabular}{lll}
        \toprule
        \textbf{Formelzeichen} & \textbf{Name} & \textbf{SI-Einheit} \\
        \midrule
        $U$ & Elektrische Spannung & \si{\volt} \\
        $R$ & Elektrischer Widerstand & \si{\ohm} \\
        $I$ & Elektrischer Strom & \si{\ampere} \\
        \bottomrule
    \end{tabular}
\end{table}

\begin{figure}[htbp]
    \centering
    \includegraphics[width=0.8\textwidth]{example-image} % dieses Bild ist Teil des mwe Pakets, welches auch ohne Befehl in der Präambel aufgerufen wird
    \caption{Beispielbild}
    \label{fig:bsp-bild}
\end{figure}

\section{Beispiele für die Nutzung von Abkürzungen und Glossareinträgen}

\begin{table}[htbp]
    \centering
    \caption{Befehle und Ausgaben für Paket \texttt{glossaries}}
    \label{tab:befehle-ausgaben-glossaries}
    \begin{tabular}{lll}
        \toprule
        Befehl & Kurzbefehl & Ausgabe \\
        \midrule
        \verb|\acrfull{epwg}| & \verb|\acf{epwg}| & \acrfull{epwg} \\
        \verb|\acrlong{epwg}| & \verb|\acl{epwg}| & \acrlong{epwg} \\
        \verb|\acrshort{epwg}| & \verb|\acs{epwg}| & \acrshort{epwg} \\
        \verb|\acrfullpl{epwg}| & \verb|\acfp{epwg}| & \acrfullpl{epwg} \\
        \verb|\acrlongpl{epwg}| & \verb|\aclp{epwg}| & \acrlongpl{epwg} \\
        \verb|\acrshortpl{epwg}| & \verb|\acspl{epwg}| & \acrshortpl{epwg} \\
        \verb|\Acrfull{epwg}| & \verb|\Acf{epwg}| & \Acrfull{epwg} \\
        \verb|\Acrlong{epwg}| & \verb|\Acl{epwg}| & \Acrlong{epwg} \\
        \verb|\Acrshort{epwg}| & \verb|\Acs{epwg}| & \Acrshort{epwg} \\
        \verb|\Acrfullpl{epwg}| & \verb|\Acfp{epwg}| & \Acrfullpl{epwg} \\
        \verb|\Acrlongpl{epwg}| & \verb|\Aclp{epwg}| & \Acrlongpl{epwg} \\
        \verb|\Acrshortpl{epwg}| & \verb|\Acspl{epwg}| & \Acrshortpl{epwg} \\
        \bottomrule
    \end{tabular}
\end{table}

Erste Verwendung von \verb|\gls{lms}| führt zu: \gls{lms}

Nächste Verwendung von \verb|\gls{lms}| führt zu: \gls{lms}

Wird ein im Glossar definierter Begriff mit bspw. \verb|\gls{moodle}| verwendet, führt dies zu folgender Ausgabe: \gls{moodle}

\clearpage
