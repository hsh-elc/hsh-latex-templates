\chapter{Test-Kapitel}

In \eqref{eq:test} sehen wir eine Gleichung.
\begin{equation}
    I_{out} = I_{in} \label{eq:test}
\end{equation}

\clearpage

\section{Beispiele für Zitate, Formeln, Tabellen und Abbildungen}
Erstmals wurde der Zusammenhang von elektrischer Spannung, elektrischem Strom und elektrischem Widerstand von Georg Simon Ohm beschrieben. \cite{Ohm.1827}

\begin{equation}
    U = R \cdot I
\end{equation}

\begin{table}[htbp]
    \centering
    \caption{Beispieltabelle}
    \label{tab:bsp-tabelle}
    \begin{tabular}{lll}
        \toprule
        \textbf{Formelzeichen} & \textbf{Name} & \textbf{SI-Einheit} \\
        \midrule
        $U$ & Elektrische Spannung & \si{\volt} \\
        $R$ & Elektrischer Widerstand & \si{\ohm} \\
        $I$ & Elektrischer Strom & \si{\ampere} \\
        \bottomrule
    \end{tabular}
\end{table}

\begin{figure}[htbp]
    \centering
    \includegraphics[width=0.8\textwidth]{example-image} % dieses Bild ist Teil des mwe Pakets, welches auch ohne Befehl in der Präambel aufgerufen wird
    \caption{Beispielbild}
    \label{fig:bsp-bild}
\end{figure}

\clearpage

\section{Blind Math Paper}
\blindmathpaper % erzeugt Blindtext mit mathematischen Formeln, kann entfernt werden

\section{Blind Document}
\blinddocument % Erzeugt ein Dokument mit Blindtext, Überschriften und Listen; kann entfernt werden
